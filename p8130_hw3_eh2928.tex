% Options for packages loaded elsewhere
\PassOptionsToPackage{unicode}{hyperref}
\PassOptionsToPackage{hyphens}{url}
%
\documentclass[
]{article}
\usepackage{lmodern}
\usepackage{amssymb,amsmath}
\usepackage{ifxetex,ifluatex}
\ifnum 0\ifxetex 1\fi\ifluatex 1\fi=0 % if pdftex
  \usepackage[T1]{fontenc}
  \usepackage[utf8]{inputenc}
  \usepackage{textcomp} % provide euro and other symbols
\else % if luatex or xetex
  \usepackage{unicode-math}
  \defaultfontfeatures{Scale=MatchLowercase}
  \defaultfontfeatures[\rmfamily]{Ligatures=TeX,Scale=1}
\fi
% Use upquote if available, for straight quotes in verbatim environments
\IfFileExists{upquote.sty}{\usepackage{upquote}}{}
\IfFileExists{microtype.sty}{% use microtype if available
  \usepackage[]{microtype}
  \UseMicrotypeSet[protrusion]{basicmath} % disable protrusion for tt fonts
}{}
\makeatletter
\@ifundefined{KOMAClassName}{% if non-KOMA class
  \IfFileExists{parskip.sty}{%
    \usepackage{parskip}
  }{% else
    \setlength{\parindent}{0pt}
    \setlength{\parskip}{6pt plus 2pt minus 1pt}}
}{% if KOMA class
  \KOMAoptions{parskip=half}}
\makeatother
\usepackage{xcolor}
\IfFileExists{xurl.sty}{\usepackage{xurl}}{} % add URL line breaks if available
\IfFileExists{bookmark.sty}{\usepackage{bookmark}}{\usepackage{hyperref}}
\hypersetup{
  pdftitle={Homework 3, P8130},
  pdfauthor={Emil Hafeez (eh2928)},
  hidelinks,
  pdfcreator={LaTeX via pandoc}}
\urlstyle{same} % disable monospaced font for URLs
\usepackage[margin=1in]{geometry}
\usepackage{color}
\usepackage{fancyvrb}
\newcommand{\VerbBar}{|}
\newcommand{\VERB}{\Verb[commandchars=\\\{\}]}
\DefineVerbatimEnvironment{Highlighting}{Verbatim}{commandchars=\\\{\}}
% Add ',fontsize=\small' for more characters per line
\usepackage{framed}
\definecolor{shadecolor}{RGB}{248,248,248}
\newenvironment{Shaded}{\begin{snugshade}}{\end{snugshade}}
\newcommand{\AlertTok}[1]{\textcolor[rgb]{0.94,0.16,0.16}{#1}}
\newcommand{\AnnotationTok}[1]{\textcolor[rgb]{0.56,0.35,0.01}{\textbf{\textit{#1}}}}
\newcommand{\AttributeTok}[1]{\textcolor[rgb]{0.77,0.63,0.00}{#1}}
\newcommand{\BaseNTok}[1]{\textcolor[rgb]{0.00,0.00,0.81}{#1}}
\newcommand{\BuiltInTok}[1]{#1}
\newcommand{\CharTok}[1]{\textcolor[rgb]{0.31,0.60,0.02}{#1}}
\newcommand{\CommentTok}[1]{\textcolor[rgb]{0.56,0.35,0.01}{\textit{#1}}}
\newcommand{\CommentVarTok}[1]{\textcolor[rgb]{0.56,0.35,0.01}{\textbf{\textit{#1}}}}
\newcommand{\ConstantTok}[1]{\textcolor[rgb]{0.00,0.00,0.00}{#1}}
\newcommand{\ControlFlowTok}[1]{\textcolor[rgb]{0.13,0.29,0.53}{\textbf{#1}}}
\newcommand{\DataTypeTok}[1]{\textcolor[rgb]{0.13,0.29,0.53}{#1}}
\newcommand{\DecValTok}[1]{\textcolor[rgb]{0.00,0.00,0.81}{#1}}
\newcommand{\DocumentationTok}[1]{\textcolor[rgb]{0.56,0.35,0.01}{\textbf{\textit{#1}}}}
\newcommand{\ErrorTok}[1]{\textcolor[rgb]{0.64,0.00,0.00}{\textbf{#1}}}
\newcommand{\ExtensionTok}[1]{#1}
\newcommand{\FloatTok}[1]{\textcolor[rgb]{0.00,0.00,0.81}{#1}}
\newcommand{\FunctionTok}[1]{\textcolor[rgb]{0.00,0.00,0.00}{#1}}
\newcommand{\ImportTok}[1]{#1}
\newcommand{\InformationTok}[1]{\textcolor[rgb]{0.56,0.35,0.01}{\textbf{\textit{#1}}}}
\newcommand{\KeywordTok}[1]{\textcolor[rgb]{0.13,0.29,0.53}{\textbf{#1}}}
\newcommand{\NormalTok}[1]{#1}
\newcommand{\OperatorTok}[1]{\textcolor[rgb]{0.81,0.36,0.00}{\textbf{#1}}}
\newcommand{\OtherTok}[1]{\textcolor[rgb]{0.56,0.35,0.01}{#1}}
\newcommand{\PreprocessorTok}[1]{\textcolor[rgb]{0.56,0.35,0.01}{\textit{#1}}}
\newcommand{\RegionMarkerTok}[1]{#1}
\newcommand{\SpecialCharTok}[1]{\textcolor[rgb]{0.00,0.00,0.00}{#1}}
\newcommand{\SpecialStringTok}[1]{\textcolor[rgb]{0.31,0.60,0.02}{#1}}
\newcommand{\StringTok}[1]{\textcolor[rgb]{0.31,0.60,0.02}{#1}}
\newcommand{\VariableTok}[1]{\textcolor[rgb]{0.00,0.00,0.00}{#1}}
\newcommand{\VerbatimStringTok}[1]{\textcolor[rgb]{0.31,0.60,0.02}{#1}}
\newcommand{\WarningTok}[1]{\textcolor[rgb]{0.56,0.35,0.01}{\textbf{\textit{#1}}}}
\usepackage{graphicx,grffile}
\makeatletter
\def\maxwidth{\ifdim\Gin@nat@width>\linewidth\linewidth\else\Gin@nat@width\fi}
\def\maxheight{\ifdim\Gin@nat@height>\textheight\textheight\else\Gin@nat@height\fi}
\makeatother
% Scale images if necessary, so that they will not overflow the page
% margins by default, and it is still possible to overwrite the defaults
% using explicit options in \includegraphics[width, height, ...]{}
\setkeys{Gin}{width=\maxwidth,height=\maxheight,keepaspectratio}
% Set default figure placement to htbp
\makeatletter
\def\fps@figure{htbp}
\makeatother
\setlength{\emergencystretch}{3em} % prevent overfull lines
\providecommand{\tightlist}{%
  \setlength{\itemsep}{0pt}\setlength{\parskip}{0pt}}
\setcounter{secnumdepth}{-\maxdimen} % remove section numbering

\title{Homework 3, P8130}
\author{Emil Hafeez (eh2928)}
\date{10/15/2020}

\begin{document}
\maketitle

\begin{Shaded}
\begin{Highlighting}[]
\KeywordTok{library}\NormalTok{(tidyverse)}
\end{Highlighting}
\end{Shaded}

\begin{verbatim}
## -- Attaching packages ----------------------------------------------------------------------------------------------------------------------- tidyverse 1.3.0 --
\end{verbatim}

\begin{verbatim}
## v ggplot2 3.3.2     v purrr   0.3.4
## v tibble  3.0.3     v dplyr   1.0.2
## v tidyr   1.1.2     v stringr 1.4.0
## v readr   1.3.1     v forcats 0.5.0
\end{verbatim}

\begin{verbatim}
## -- Conflicts -------------------------------------------------------------------------------------------------------------------------- tidyverse_conflicts() --
## x dplyr::filter() masks stats::filter()
## x dplyr::lag()    masks stats::lag()
\end{verbatim}

\begin{Shaded}
\begin{Highlighting}[]
\NormalTok{exercise_df =}
\StringTok{  }\KeywordTok{read_csv}\NormalTok{(}
      \StringTok{"./data/exercise.csv"}\NormalTok{)}
\end{Highlighting}
\end{Shaded}

\begin{verbatim}
## Parsed with column specification:
## cols(
##   Group = col_double(),
##   Age = col_double(),
##   Gender = col_double(),
##   Race = col_double(),
##   HTN = col_double(),
##   T2DM = col_double(),
##   Depression = col_double(),
##   Smokes = col_double(),
##   Systolic_PRE = col_double(),
##   Systolic_POST = col_double()
## )
\end{verbatim}

\hypertarget{problem-1}{%
\section{Problem 1}\label{problem-1}}

For each question, make sure to state the formulae for hypotheses,
test-statistics, decision rules/p-values, and provide interpretations in
the context of the problem. Use a type I error of 0.05 for all tests.

\hypertarget{problem-1.a}{%
\subsubsection{Problem 1.a}\label{problem-1.a}}

Perform appropriate tests to assess if the Systolic BP at 6 months is
significantly different from the baseline values for the intervention
group.

********Check for normality, consider a bonferroni adjustment, state
that it's a a paired two-tailed test using an estimate. Then provide the
formulas for the test statistic needed, critical value needed, the
value, and then get these. and then interpret in context.********

H0, HA, mu1, mu2, test statistic formula, critical value needed,
interpret

\hypertarget{problem-1.a.i}{%
\paragraph{Problem 1.a.i}\label{problem-1.a.i}}

In order to determine an appropriate test, we check for normality using
visual inspection of the plot of systolic blood pressure in the
Intervention group at both baseline and endline, using raw data. This is
explored in Problem 1.c.i and utilizes mean \(\mu_{pre}\) and
\(\mu_{post}\) respectively.

We first consider the changes in the intervention group between Baseline
and Endline first. We determine, given that these are the same patients
with data collected at two different timepoints and we do not have
reason to test a specific directionality, to use a two-sided Paired
t-test.

The \(H_0\) is that \(\mu_{pre} -\mu_{post} = 0\) or \(\Delta = 0\). The
\(H_A\) is \(\mu_{pre} -\mu_{post} \neq 0\) or \(\Delta \neq 0\).

The test statistic is \(t=\frac{\overline d-0}{s_{d} / \sqrt{n}}\) where
\(\overline d\) is the point estimate of the mean difference,
\(s_{d} / \sqrt{n}\) is the estimated standard error of the differences,
and we use the critical value of \(t_{n-1,1-{\alpha/2}}\). We could use
a Bonferroni correction or Tukey's, considering that we will be
implementing multiple significance tests, but we say it is not necessary
for the case of this homework problem.

Using \(t=\frac{\overline d-0}{s_{d} / \sqrt{n}}\) =
\(t=\frac{-8.58 -0}{17.17/ \sqrt{36}}\) =
\(t=\frac{-8.58}{17.17/ \sqrt{36}}\) \(\approx -2.99825\). The critical
value is given the the percentile of the t distribution with (n-1)
degrees of freedom, \texttt{qt(0.975,35)} \(\approx\) \(2.03\), such
that we find evidence to reject the null hypothesis and conclude that in
the intervention group, the mean systolic blood pressure at Endline is
significantly different than the mean systolic blood pressure at
Baseline.

\hypertarget{problem-1.a.ii}{%
\paragraph{Problem 1.a.ii}\label{problem-1.a.ii}}

Similarly, in order to determine an appropriate test, we check for
normality using visual inspection of the plot of systolic blood pressure
in the Control group at both Baseline and Endline, using raw data. This
is explored in Problem 1.c.i and utilizes mean \(\mu_{pre}\) and
\(\mu_{post}\) respectively.

We determine, given that these are the same patients with data collected
at two different timepoints and we do not have reason to test a specific
directionality, to use a two-sided Paired t-test.

The \(H_0\) is that \(\mu_{pre} -\mu_{post} = 0\) or \(\Delta = 0\). The
\(H_A\) is \(\mu_{pre} -\mu_{post} \neq 0\) or \(\Delta \neq 0\).

The test statistic is \(t=\frac{\overline d-0}{s_{d} / \sqrt{n}}\) where
\(\overline d\) is the point estimate of the mean difference,
\(s_{d} / \sqrt{n}\) is the estiamted standard error of the differences,
and we use the critical value of \(t_{n-1,1-{\alpha/2}}\). We could use
a Bonferroni correction, Tukey's or others, considering that we will be
implementing multiple significance tests, but we say it is not necessary
for the case of this homework problem.

Using \(t=\frac{\overline d-0}{s_{d} / \sqrt{n}}\) =
\(t=\frac{-3.33 -0}{14.81/ \sqrt{36}}\) \(\approx - 1.3491\). The
critical value is given the the percentile of the t distribution with
(n-1) degrees of freedom, \texttt{qt(0.975,35)} \(\approx\) \(2.03\),
such that we fail to reject the null hypothesis and conclude that in the
Control group, the mean systolic blood pressure at Endline is not
significantly different than the mean systolic blood pressure at
Baseline.

\hypertarget{problem-1.b}{%
\paragraph{Problem 1.b}\label{problem-1.b}}

Now, we assess the systolic blood pressure absolutely changes between
the two groups using an independent, two-sampled t-test. We assume
independence based on the given information on the sampling mechanism.
Since we do not know the two population variances, we first check for
equality of variances.

The statistic is given by
\(F=\frac{s_{1}^{2}}{s_{2}^{2}} \sim F_{n_{1}-1, n_{2}-1}\), where the
null hypothesis is that \(\sigma^2_1 = \sigma^2_2\) and the alternate
hypothesis is \(\sigma^2_1\neq \sigma^2_2\).

We reject the null hypothesis when or \$ F\textless F\_\{n\_\{1\}-1,
n\_\{2\}-1, a / 2\}\$ and fail to reject the null when
\(F_{n_{1}-1, n_{2}-1, \alpha / 2} \leq F \leq F_{n_{1}-1, n_{2}-1,1-\alpha / 2}\).

In our case, we use \$F = \frac{14.81^2}{17.17^2} = 0.74399 \$ and the
critical value is \texttt{qf(0.975,\ 35,\ 35)} = \(1.961\), such that we
fail to reject the null hypothesis and conclude that the variance of
sample 1 is not significantly different from the variance of sample 2.

\hypertarget{problem-1.c}{%
\subsubsection{Problem 1.c}\label{problem-1.c}}

\hypertarget{problem-1.c.i}{%
\paragraph{Problem 1.c.i}\label{problem-1.c.i}}

\begin{Shaded}
\begin{Highlighting}[]
\CommentTok{#plot of intervention at baseline to examine normality}
\NormalTok{exercise_df }\OperatorTok
\StringTok{  }\KeywordTok{ggplot}\NormalTok{(}\KeywordTok{aes}\NormalTok{(}\DataTypeTok{x =}\NormalTok{ Systolic_PRE)) }\OperatorTok{+}
\StringTok{  }\KeywordTok{geom_histogram}\NormalTok{(}\DataTypeTok{binwidth =} \DecValTok{8}\NormalTok{, }\DataTypeTok{fill =} \StringTok{"black"}\NormalTok{, }\DataTypeTok{colour =} \StringTok{"black"}\NormalTok{, }\DataTypeTok{alpha =} \FloatTok{0.9}\NormalTok{) }\OperatorTok{+}
\StringTok{  }\KeywordTok{labs}\NormalTok{(}
    \DataTypeTok{x =} \StringTok{"Systolic Blood Pressure at Baseline"}\NormalTok{,}
    \DataTypeTok{y =} \StringTok{"Density"}\NormalTok{,}
    \DataTypeTok{title =} \StringTok{"Histogram and Density Curve for the Original Normal Data"}\NormalTok{) }\OperatorTok{+}
\StringTok{  }\KeywordTok{scale_fill_viridis_d}\NormalTok{(}\StringTok{""}\NormalTok{) }\OperatorTok{+}
\StringTok{  }\KeywordTok{theme}\NormalTok{(}\DataTypeTok{legend.position =} \StringTok{"none"}\NormalTok{)}
\end{Highlighting}
\end{Shaded}

\includegraphics{p8130_hw3_eh2928_files/figure-latex/unnamed-chunk-2-1.pdf}

\begin{Shaded}
\begin{Highlighting}[]
\CommentTok{#plot of intervention at endline to examine normality}
\NormalTok{exercise_df }\OperatorTok
\StringTok{  }\KeywordTok{ggplot}\NormalTok{(}\KeywordTok{aes}\NormalTok{(}\DataTypeTok{x =}\NormalTok{ Systolic_POST)) }\OperatorTok{+}
\StringTok{  }\KeywordTok{geom_histogram}\NormalTok{(}\DataTypeTok{binwidth =} \DecValTok{8}\NormalTok{, }\DataTypeTok{fill =} \StringTok{"black"}\NormalTok{, }\DataTypeTok{colour =} \StringTok{"black"}\NormalTok{, }\DataTypeTok{alpha =} \FloatTok{0.9}\NormalTok{) }\OperatorTok{+}
\StringTok{  }\KeywordTok{labs}\NormalTok{(}
    \DataTypeTok{x =} \StringTok{"Systolic Blood Pressure at Baseline"}\NormalTok{,}
    \DataTypeTok{y =} \StringTok{"Density"}\NormalTok{,}
    \DataTypeTok{title =} \StringTok{"Histogram and Density Curve for the Original Normal Data"}\NormalTok{) }\OperatorTok{+}
\StringTok{  }\KeywordTok{scale_fill_viridis_d}\NormalTok{(}\StringTok{""}\NormalTok{) }\OperatorTok{+}
\StringTok{  }\KeywordTok{theme}\NormalTok{(}\DataTypeTok{legend.position =} \StringTok{"none"}\NormalTok{)}
\end{Highlighting}
\end{Shaded}

\includegraphics{p8130_hw3_eh2928_files/figure-latex/unnamed-chunk-3-1.pdf}

DISCUSS \#\#\#\# Problem 1.c.ii

\end{document}
